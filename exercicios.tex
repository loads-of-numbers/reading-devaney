% Diário de leitura de "An Introduction to Chaotic Dynamical Systems (3rd edition)", de Robert L. Devaney. Contém anotações sobre conceitos importantes, resoluções de exercícios e qualquer outra forma de texto que eu julgar importante ou interessante; mas, principalmente, exercícios. Por ser de uma natureza mais pessoal, tudo está um pouco fora de ordem.

\documentclass[11pt]{article}

%% ----- Import Preamble -----
\usepackage{preamble}
\usepackage{accents}

%% Document

\begin{document}
\setcounter{section}{2}
\section{Elementary definitions}
\subsection{Exercícios}
\textbf{(1/2) a)} Para todo $x_0 \in \mathbb{R}$, converge para o número de Dottie $D = 0,739085133...$ — o único ponto fixo de $f(x) = \cos x$.

\textbf{b)} Para todo $x_0 \in \mathbb{R}$, converge — muito lentamente — para o único ponto fixo: 0.

\textbf{c)} Sem ponto fixo. Diverge para $+\infty$.

\textbf{d)} Para $x < 1$, converge para 1, o único ponto fixo. Para $x > 1$, diverge para $+\infty$.

\textbf{e)} Para todo $x_0 \in \mathbb{R}$, converge — muito lentamente — para o único ponto fixo: 0. \qed

\textbf{(3/4/5)} Muita coisa para escrever. Fiz em rascunhos. \qed

\textbf{(6)} \textcolor{red}{Concluído, redigir.}

\textbf{(7)} Se $f$ é homeomorfismo de $\mathbb{R}$, deve ser uma função monótona estrita (pois é injetora e contínua). Sejam $x_0 < x_1 < \cdots < x_n$ os pontos da órbita periódica de período primo $n > 2$; note que $f$ aplicada a tal órbita é uma bijeção $\{x_0, \cdots, x_n\} \to \{x_0, \cdots, x_n\}$. Caso $f$ seja decrescente, $f(x_0) > f(x_1) > \cdots > f(x_n)$ e, como $f(x_0)$ é maior que todos os outros elementos, $f(x_0) = x_n$. Analogamente, $f(x_n)$ é menor que os outros, então $f(x_n) = x_0$. Disso, já tiramos uma contradição, pois $x_0 \to x_n \to x_0$ forma um ciclo fechado sem os elementos restantes; impossível.

Caso $f$ seja crescente, mudemos levemente a notação: seja $x_0 \in \mathbb{R}$ um valor inicial da órbita periódica e $x_k = f^k(x_0)$. Assim, como $f(x_0) \neq x_0$, $x_0 < x_1$ ou $x_0 > x_1$; por indução, pois $x_{n} = f(x_{n-1})$, temos $x_0 < x_1 < \cdots < x_n$ ou $x_0 > x_1 > \cdots > x_n$. Mas sabemos que $x_n = x_0$, e chegamos numa contradição em ambos os casos.

Para o exemplo pedido, basta tomar $g(x) = -x$ e verificar que ela satisfaz as condições. \qed

\textbf{(8)} Suponha $f$ uma função com um ponto eventualmente periódico $x_0 \in \mathbb{R}$, de período primo $n$; como $x_0$ não é periódico, existe um menor $m \in \mathbb{N}$ tal que $f^{m + n}(x_0) = f^m(x_0)$, $m \neq 0$. Portanto, $m - 1 \in \mathbb{N}$. Note, assim, que $f(f^{m-1}(x_0)) = f^m(x_0) = f(f^{m + n - 1}(x_0))$, mas $f^{m-1}(x_0) \neq f^{m+n-1}(x_0)$ porque, senão, $m$ não seria o menor natural com a propriedade definida. Assim, $f$ não pode ser injetora e, em particular, não pode ser um homeomorfismo.

É interessante observar que esse argumento é facilmente generalizado para uma injeção em um espaço topológico qualquer. \qed

\textbf{(9)} Identificando $S^1$ com $\mathbb{R}/2\pi\mathbb{Z}$, temos que $S$ é um mapa cujo levantamento para $\mathbb{R}$ é dado por $s(x) = x + \omega + \epsilon\sin(x)$, onde $S(\theta) = s(\theta) \mod 2\pi$. Note que, se provarmos que $s$ é injetora e sua imagem mod $2\pi$ é sobrejetora em $[0,2\pi)$, teremos que $S$ é uma bijeção.

Para tal, temos $s'(x) = 1 + \epsilon\cos(x) \geq 1 - |\epsilon| > 1 - 1 = 0$ (pois $|\epsilon| < 1$ e $|\cos(x)| \leq 1$), implicando que $s$ é estritamente crescente; de fato, injetora. Além disso, observe que $s(0) = \omega$ e $s(2\pi) = \omega + 2\pi$ e, como $s$ é contínua, o teorema do valor intermediário nos garante que $[\omega, \omega + 2\pi] \subset s([0,2\pi])$. Mas é claro que $\{x$ mod $2\pi\ |\ x \in [\omega, \omega + 2\pi]\} = [0,2\pi]$ e, assim, $s$ restrita e induzida a $[0,2\pi)$ é sobrejetora também. Juntando tudo, temos o que queríamos.

Por fim, como $S^1$ é compacto e Hausdorff, temos que $S$ é uma bijeção contínua de um espaço compacto a um espaço Hausdorff. Então, é automaticamente um homeomorfismo. \qed

\textbf{(10)} Identifiquemos $S^1$ como $[0, 2\pi)$; como visto no livro, para o mapa dado, $x \in \text{Per}_n(f) \iff x = 2\pi k/(2^n - 1),\ 0 \leq k \leq 2^n - 2$, ou seja, $x$ representa uma $(2^n - 1)$-ésima raiz da unidade. Então, defina $S = \bigcup_{n=1}^{\infty} \text{Per}_n(f)$, isto é, o conjunto de todos os pontos periódicos de $f$, independente do período. 

Tome $x_0 \in [0, 2\pi)$ qualquer, $\epsilon > 0$ e seja $I := (x_0 - \epsilon, x_0 + \epsilon) \subset \mathbb{R}$. Sabemos que as raízes da unidade são distribuídas uniformemente em $S^1$, e, para um $n$ qualquer, a distância de uma para a próxima é $L_n = 2\pi (k+1)/(2^n - 1) - 2\pi k/(2^n - 1) = 2\pi/(2^n - 1)$; sabemos que $L_n \to 0$ quando $n \to \infty$. Portanto, tome $m \in \mathbb{N}$ t.q. $L_m < 2\epsilon$ e, assim, é fácil ver que, com $2^m - 1$ raízes distribuídas uniformemente em $[0, 2\pi)$ e uma distância de $L_m$ dentre elas, deve haver ao menos uma raíz em $I$.

Concluímos que $S$ é denso em $S^1$. \qed

\textbf{(11)} Argumento essencialmente idêntico ao anterior, pois um ponto é eventualmente fixo para esse mapa sse é da forma $\theta_k = 2\pi k/2^n$ para algum $n \in \mathbb{N}$ e $0 \leq k \leq 2^n - 1$. \qed

\section{Hyperbolicity}
\subsection{Exercícios}
\textbf{(1)} Nos rascunhos. \qed

\textbf{(2)} \textcolor{red}{Feito até (b).} \qed

\textbf{(3)} Suponha que existe uma sequência de pontos fixos hiperbólicos $(x_n)_{n\in\mathbb{N}}$ que converge para $p$, outro ponto fixo distinto dos $x_n$. Como $f$ é difeomorfismo, $f'(p)$ existe e
\[f'(p) = \lim_{x\,\to\,p}{\frac{f(x) - f(p)}{x - p}}.\]
Já que $x_n \to p$ e $x_n \neq p\ \forall n \in \mathbb{N}$, pelo teorema de Heine,
\[\lim_{x\,\to\,p}{\frac{f(x) - f(p)}{x - p}} = \lim_{n\to\infty}{\frac{f(x_n) - f(p)}{x_n-p}} = \lim_{n\to\infty}{\frac{x_n - p}{x_n-p}} = 1,\]
o que contradiz a hiperbolicidade de $p$. Ou seja, os pontos fixos hiperbólicos de qualquer difeomorfismo são isolados.

Agora, finalizamos observando que $f$ é homeomorfismo em $\mathbb{R}$, então seus pontos periódicos são de período 1 ou 2 apenas. Além disso, é claro que $\text{Per}_2(f) = \text{Fix}(f^2)$ e que $\text{Fix}(f) = \text{Per}_1(f) \subset \text{Per}_2(f)$. Portanto, como composição de difeomorfismos é difeomorfismo, $f^2$ é difeomorfismo e analisar seus pontos fixos implica em analisar todos os pontos periódicos de $f$. Isto é, o resultado dos parágrafos acima garante que os pontos periódicos de $f$ são isolados. \qed

\textbf{(4)} Defina o \textit{tent map} $T\colon [0,1] \to [0,1]\colon$
\[T(x) = \begin{cases}
    2x,\ 0 \leq x \leq 1/2 \\
    2(1-x),\ 1/2 \leq x \leq 1
\end{cases}\]
Para cada $n \in \mathbb{N}$, tome um $1/2^n \leq p \leq 1/2^{n-1}$ qualquer; então, $0 \leq 2^m p \leq 1/2^{n-m-1} < 1/2$ para todo $m < n - 1$, mas $1/2 \leq 2^{n-1}p \leq 1$. Assim, compondo $n-1$ vezes a função $2x$ e a função $2(1 - x)$ uma vez, nessa ordem, temos
\[T^n(p) = 2(1 - 2^{n-1}p).\]
Também note que
\[T^n(p) = p \iff p = \frac{2}{2^n + 1}.\]
Isto é, para cada $n \in \mathbb{N}$, $p_n = 2/(2^n + 1)$ é um ponto de período $n$ de $T$. É claro que $p_n \to 0$ quando $n \to \infty$ e, além disso, se $n > 0$,
\[\left|(T^n)'(p)\right| = \left|-2^n\right| = 2^n \neq 1,\]
então todo $p_n$ é hiperbólico. Concluímos observando que $0$ é um ponto fixo de $T$ e $f'(0) = 2 \neq 1$; ou seja, $0$ é um ponto não-isolado no conjunto dos pontos periódicos hiperbólicos. \qed

\textbf{(5)} \textcolor{red}{Fazer.} \qed

\textbf{(6)} Buscando por pontos fixos, $x^3 - \alpha x = x \iff x[x^2 - (\alpha + 1)] = 0$; caso $\alpha \in (-1,1]$, $f_\alpha$ tem três pontos fixos — $x_0 = 0,\ x_{\pm} = \pm\sqrt{\alpha + 1}$ — e, caso $\alpha \in (-\infty, -1]$, $f_\alpha$ tem apenas um ponto fixo ($x_0 = 0$). Portanto, $\alpha = -1$ é um ponto de bifurcação. 

Além disso, $f_\alpha'(x) = 3x^2 - \alpha$, então $f'_\alpha(x_0) = -\alpha$ e $f'(x_\pm) = 2\alpha + 3$. Analisando as derivadas, concluímos que os retratos de fase são da seguinte forma (aqui, R indica um ponto repulsor, A indica um ponto atrator e N indica um ponto não-hiperbólico):
\begin{equation*}
    \begin{aligned} 
        & \boxed{\alpha \in (-1,1)}: (x_{-},x_0,x_{+}) = (\text{R, A, R}) \\
        & \boxed{\alpha \in (-\infty,-1)}: (x_0) = (\text{R})
    \end{aligned}
    \ \ \ \ \ \ \ 
    \begin{aligned}
        & \boxed{\alpha = 1}: (x_{-}, x_0, x_{+}) = (\text{R, NA, R}) \\
        & \boxed{\alpha = -1}: (x_0) = (\text{NR})
    \end{aligned}
\end{equation*}

Analisou-se a atração ou repulsão dos pontos não-hiperbólicos observando o seguinte: $|f_{-1}(x)| = |x + x^3| > |x|$ implica que pontos próximos de $x_0$ se afastarão dele (aumentando de magnitude). Semelhantemente, $|f_1(x)| = |x^3 - x| \leq |x| - |x^3| \leq |x|$ para $x \in [-1,1]$, então pontos próximos de $x_0$ se aproximarão dele (diminuindo de magnitude).

Por fim, para $\alpha > 1$, observe que $f'_\alpha(x_0) < -1$ e, portanto, $x_0$ se torna repulsor. Portanto, $\alpha = 1$ também é um ponto de bifurcação. Variando o $\alpha$ e buscando por outros pontos não-hiperbólicos através das derivadas nos pontos fixos, não encontra-se nenhum outro no intervalo dado; conclui-se que $\alpha_{\pm} = \pm1$ são os únicos pontos de bifurcação que nos interessam. \qed

\textbf{(7)} Se $f_k(x) = kx$, então o único ponto fixo é $x = 0$ e $f_k'(0) = k$. Se $k > 1$, a origem é repulsora; se $0 < k < 1$, a origem é atratora; se $-1 < k < 0$, a origem é atratora; e, se $k < -1$, a origem é repulsora. Assim, temos nossos quatro conjuntos abertos para valores de $k$: $(-\infty,-1),\ (-1,0),\ (0,1),\ (1,\infty)$.

Sobre os casos excepcionais: em $f_0(x) = 0$, todo ponto é eventualmente fixo. Em $f_1(x) = x$, todo ponto é fixo. E, em $f_{-1}(x) = -x$, todo ponto (exceto o 0) é de período primo 2. \qed

\section{An example: the quadratic family}
\subsection{Caracterização do conjunto de Cantor}
Acredito que o resultado a seguir é interessante por si só para merecer a própria seção; contudo, ele também nos auxiliará com alguns exercícios mais adiante.

\begin{lemma}\label{lem:cantor_ternary}
    Seja $\mathcal{C}$ o conjunto ternário de Cantor. Então, $p \in \mathcal{C} \iff p = 0.X_1X_2X_3\dots_3$, onde $X_i \in \{0,2\}\ \forall i \in \mathbb{N}$ — isto é, $\mathcal{C}$ é composto exatamente dos pontos em $[0,1]$ que admitem uma representação apenas de $0$s e $2$s em base $3$.
\end{lemma}

\begin{proof}
    Tome 
    \[p = 0.X_1X_2X_3\dots X_k \dots_3 = \sum^{\infty}_{k=1}{\frac{X_k}{3^k}},\]
    como acima. Suponha que existe um menor $n \in \mathbb{N}$ tal que $X_n = 1$ e os próximos dígitos não são apenas 0s ou apenas 1s (senão, $p$ é da forma $0.000\ldots0111\ldots = 0.000\ldots02$ ou $0.000\ldots01 = 0.000\ldots0222\ldots$, ambos podendo ser representados apenas com $0$s e $2$s); provaremos que todos os $p$ sob essas condições formam exatamente $\text{I} \setminus \mathcal{C}$. \textcolor{red}{Finalizar.}
\end{proof}

\subsection{Exercícios}
\textbf{(1)} Idêntico à primeira parte da demonstração da proposição 5.3 no livro. \qed

\textbf{(2)} Note que $F(x) = x \iff \mu x(1-x) = x \iff x[\mu x - (\mu - 1)] = 0$, então $x_1 = 0$ ou $x_2 = \frac{\mu - 1}{\mu}$ são os pontos fixos de $F$. Caso $0 < \mu \leq 1$, $x_2 \leq 0$; mas $F(x) \geq 0$ para $x \in \text{I}$, então nenhuma órbita com valor inicial em I pode convergir para $x_2$ (a não ser que ele seja nulo, $x_1 = x_2$). Além disso, observe que $F(x) = \mu x (1-x) \leq x(1-x) \leq x$, pois $\mu \in (0,1]$ e $1-x \leq 1$ para $x \in \text{I}$; isto é, as órbitas com valores iniciais em I são decrescentes.

Como órbitas monótonas em conjuntos compactos devem convergir para algum ponto fixo, $F^n(x)$, $x$ em I, deve convergir para o único ponto fixo disponível: $x_1 = 0$. \qed

\textbf{(8)} Seja $\mathcal{C}$ o conjunto ternário de Cantor e sejam $A_0,\ A_1,\ A_2, \ldots$ as uniões de intervalos abertos retiradas a cada etapa de construção. Assim,
\[\mathcal{C} = \text{I}\ \setminus \left(\bigcup^{\infty}_{n=0}A_n\right).\]
\underline{Fechado}: Pelas leis de De Morgan,
\[\mathcal{C} = \text{I}\ \cap \bigcap^{\infty}_{n=0}A_n^c,\]
que é uma intersecção de conjuntos fechados e, portanto, fechada. 

\underline{Não-vazio}: $0 = 0.000\ldots$ é representado apenas por 0s em base 3, então o lema \ref{lem:cantor_ternary} garante que $0 \in \mathcal{C}$.

\underline{Perfeito}: Se $p \in \mathcal{C}$, represente-o como 0s e 2s em base 3 e tome a sequência $(p_n)_{n \in \mathbb{N}}$ onde $p_n$ possui os mesmos dígitos que $p$ exceto pelo n-ésimo dígito, que é trocado (0 é trocado por 2 e 2 é trocado por 0). É claro que $p_n \neq p$ e $p_n \in \mathcal{C}$ para todo $n \in \mathbb{N}$; além disso, como $|p_n - p| = 2/3^n \to 0$ quando $n \to \infty$, então $p_n \to p$. Portanto, todo $p \in \mathcal{C}$ é ponto de acumulação.

\underline{Totalmente desconexo}: Caso existisse um intervalo $[x,y] \subset \mathcal{C}$, teríamos que a medida $L$ de $\mathcal{C}$ seria, ao menos, $L \geq |x-y| > 0$, contradizendo o resultado do exercício (9) abaixo. \qed

\textbf{(9)} Na n-ésima etapa de construção de $\mathcal{C}$, retira-se $2^{n-1}$ intervalos, de comprimento $1/3^n$ cada, do conjunto da etapa anterior. Isto é, denotando por $L_n$ a soma dos comprimentos na n-ésima etapa:
\[L_n = L_{n-1} - \frac{2^{n-1}}{3^n}.\]
Como $L_0 = 1$, concluímos indutivamente que
\begin{align*}
    L_n = & L_{n-1} - \frac{2^{n-1}}{3^n} \\
    & = L_{n-2} - \frac{2^{n-2}}{3^{n-1}} - \frac{2^{n-1}}{3^n} \\
    & = \cdots \\
    & = L_0 - \sum^{n-1}_{k=0}{\frac{2^k}{3^{k+1}}},
\end{align*} 
isto é,
\[\boxed{L_n = 1 - \frac{1}{3}\sum^{n-1}_{k=0}{\left(\frac{2}{3}\right)^k}}\]

Observe que o termo subtraído na expressão é uma soma geométrica que, quando $n \to \infty$, se torna uma série geométrica. No caso,
\[\sum^{\infty}_{k=0}{\left(\frac{2}{3}\right)^k} = \frac{1}{1 - 2/3} = 3\]
e, portanto, $L_n \to L_\infty = 1 - \frac{1}{3}\cdot3 = 1-1 = 0$. \qed

\textbf{(10)} O processo é análogo ao do exercício anterior. Contudo, agora, os intervalos restantes após a n-ésima etapa possuem $(2/5)^n$ de comprimento cada e, assim, o comprimento a ser retirado para a (n+1)-ésima etapa é de $2^n \cdot 1/5 \cdot (2/5)^n = 4^{n}/5^{n+1}$. Portanto,
\[\boxed{L_n = 1 - \frac{1}{5}\sum^{n-1}_{k=0}\left(\frac{4}{5}\right)^k}\]
e, assim,
\[L_\infty = 1 - \frac{1}{5}\cdot\frac{1}{1 - 4/5} = 1 - \frac{1}{5}\cdot5 = 0,\]
da mesma forma que no exercício anterior. Interessantemente, o processo pode ser generalizado para qualquer conjunto de Cantor onde se retira uma fração fixa $\alpha$ ($0 < \alpha < 1$) do meio dos subintervalos e, ainda assim, o comprimento final sempre será $L_\infty = 0$. \qed

\textbf{(11)} Nesse exercício, por conveniência, a discussão será feita exclusivamente em base 3, onde $3_{10} = 10_3$ e uma multiplicação por $3$ equivale a um deslocamento da vírgula. $L\colon \mathcal{C} \cap [0,1/3] \to \mathcal{C}$ é escrita como $L(x) = 10x$; é claro que $L$ é contínua e, além disso, é fácil verificar que possui inversa $L^{-1}\colon \mathcal{C} \to \mathcal{C} \cap [0,1/3]$ dada por $L^{-1}(x) = x/10$, que também é claramente contínua. Basta mostrar que as imagens de $L$ e $L^{-1}$ estão, de fato, nos respectivos contradomínios.

Um ponto $x$ pertence a $\mathcal{C} \cap [0,1/3]$ se e somente se está em $\mathcal{C}$ e seu primeiro dígito (na representação com 0s e 2s) após a vírgula é 0, pois $x \leq 1/3$; ou seja, $x = 0.0X_1X_2\ldots$, com $X_i \in \{0,2\}$. Portanto, se $p = 0.Y_1Y_2Y_3\ldots$ pertence a $\mathcal{C}$, temos que $L^{-1}(p) = 1/10 \cdot 0.Y_1Y_2Y_3\ldots = 0.0Y_1Y_2\ldots \in \mathcal{C} \cap [0,1/3]$, como queríamos — já que os dígitos continuam sendo 0 ou 2 e o primeiro dígito é sempre 0. Semelhantemente, $L(0.0Y_1Y_2\ldots) = 0.Y_1Y_2\ldots \in \mathcal{C}$ para todo $0.0Y_1Y_2\ldots \in \mathcal{C} \cap [0,1/3]$. Assim, as funções estão bem-definidas e $L$ é um homeomorfismo. \qed

\textbf{(12)} \textcolor{red}{Fazer.}
\end{document}